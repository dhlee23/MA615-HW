\documentclass[13pt, letterpaper]{article}
\usepackage{layout}
\addtolength{\hoffset}{-3.3cm}
\addtolength{\voffset}{-2.5cm}
%\addtolength{\textwidth}{1cm}
%\addtolength{\textwidth}{1cm}
%\addtolength{\textwidth}{1cm}
\pdfpageheight=22cm
\pdfpagewidth=15cm


\begin{document}
Extract from:

Bradley Efron and Trevor Hastie

\emph{Computer Age Statistical Inference: Algorithms, Evidence, and Data Science}

\emph{Cambridge University Press, 2016}

$https://web.stanford.edu/{\sim}hastie/CASIfiles/PDF/casi.pdf$ \\\\\\\\\

Modern Bayesian practice uses various strategies to construct an appropriate “prior” $ g(\mu) $ in the absence of prior experience, leaving many statisticians unconvinced by the resulting Bayesian inferences. Our second example illustrates the difficulty.\\
\\
Table 3.1 \textit{Scores from two tests taken by 22 students}, \textbf{mechanics} \emph{and} \textbf{vectors}.\\\\\
\begin{tabular}{llllllllllll}
                                    & 1  & 2  & 3  & 4  & 5  & 6  & 7  & 8  & 9  & 10 & 11 \\ \hline
\textbf{mechanics}    & 7  & 44 & 49 & 59 & 34 & 46 & 0  & 32 & 49 & 52 & 44 \\
\textbf{vectors}          & 51 & 69 & 41 & 70 & 42 & 40 & 40 & 45 & 57 & 64 & 61 \\ \hline
\end{tabular}
\\\\\\
\begin{tabular}{llllllllllll}
                                     & 13  & 13  & 14  & 15  & 16  & 17  & 18  & 19  & 20 & 21 & 22\\ \hline
\textbf{mechanics}     & 7  & 44 & 49 & 59 & 34 & 46 & 0  & 32 & 49 & 52 & 44 \\
\textbf{vectors}          & 51 & 69 & 41 & 70 & 42 & 40 & 40 & 45 & 57 & 64 & 61 \\ \hline
\end{tabular}\\\\

Table 3.1 shows the scores on two tests, mechanics and vectors, achieved by $n=22$ students. The sample correlation coefficient between the two scores is $\hat{\theta} = 0.498$,
$$\hat{\theta} = 
\sum_{i=1}^{22}(m_{i}-\bar{m})(v_{i}-\bar{v})\;
\Bigg/
\left[
\sum_{i=1}^{22}(m_{i}-\bar{m})^2\sum_{i=1}^{22}(v_{i}-\bar{v})^2
\right]^{1/2}
$$
with $m$ and $v$ short for mechanics and vectors, $\bar{m}$ and $\bar{v}$ their averages.

\end{document}